\documentclass[a4paper]{article}

\usepackage[english]{babel}
\usepackage[utf8]{inputenc}
\usepackage{amsmath}
\usepackage{graphicx}
\usepackage{parskip}
\usepackage{amssymb}
\usepackage{mathtools}
\usepackage{svg}
\usepackage{pdfpages}
\usepackage{lscape}
\usepackage{enumerate}

\title{SE464 Lab 3: Architectural Analysis of Puppet}
\author{Shi Shi, Hong Wen Zhu, Dane Carr, Jaier Wang, Lucas Wojciechowski}
\date{\today}

\begin{document}

\maketitle

\section{System  Functionality} % (5  marks)

% State the purpose of  the system. Describe  its functionality by  listing 3-7
% main  user  stories that  it  supports. For the user  stories,  use the
% following format: "As  a <user role>,  I want  <goal/desire> so  that
% <benefit>"

Puppet is a system which allows a software engineer to elegantly describe the desired state of a system and use that description to generate a deployment plan and monitor the deployed systems.

More formally, as user stories:
\begin{itemize}
\item As a software engineer, I want a powerful and consistent way to describe my infrastructure deployment plan so that it is thourough and easy to maintain.
\item As a software engineer, I want a way to ensure my infrastructure is always in the correct state so that it is robust and .
\item As a software engineer, I want an automated way to deploy software to my infrastructure so that it takes minimal time to do a deployment and deployments can be reliably repeated.
\end{itemize}

\section{Quality Attributes and Scenarios} % (5  marks)

% Make  a list  of  the key quality attributes  that  the system  must  satisfy.
% Document  each  attribute with  one or more  quality attribute analysis
% scenario (see  the lecture material  on  quality attributes  for examples).

\subsection{Robust}

As a ``metasystem" designed to make other systems robust, Puppet must be robust itself.

\subsection{Extensible}

As a general purpose deployment tool, Puppet must be extensible enough to accomadate many different types of systems.

\subsection{Easy to Use}

As a user interface to describe a complex process, Puppet must be

\subsection{Easy to Maintain}


\section{Architectural Views} % (10 marks)

% Provide a list  of  architectural views that  are relevant  for the system
% and diagram or  another adequate representation  (e.g.,  a list  or  code
% fragments)  of  each  view, along with  a short description.

% Examples  of  views that  may be  relevant  are layered view, deployment
% view, module  view  (e.g.,  file  or component packaging), behavioral  view,
% and representation  of  specific  mechanisms  (e.g.,  the intent mechanism in
% Android or  dependency  injection in  Spring).

\section{Architectural Analysis} % (10 marks)

% For each  quality attribute analysis  scenario, briefly discuss which design
% tactics and architectural styles or  patterns  (not  necessarily limited to
% those covered in  the lectures) have  been  applied to  address the attribute
% and relate  the tactics,  patterns  or  styles  to  the relevant
% architectural views.

\section{Key Weaknesses} % (5  marks)

% Identify  and briefly discuss the main  weaknesses  of  the architecture. How
% could they  be  addressed?

\section{References} % (1  mark)

% List  the references  to  the resources (books, papers, online  articles,
% etc.) you used  to  prepare this  report

http://puppetlabs.com/puppet/what-is-puppet

\end{document}
